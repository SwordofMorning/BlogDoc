\documentclass[UTF8]{ctexart}

\usepackage{ctex}
\usepackage{times}
\usepackage{geometry}
\usepackage{subcaption}
\usepackage{textcomp}
\usepackage{tocbibind}
\usepackage{fancyhdr}
\usepackage{color}
\usepackage{graphicx}
\usepackage{titling}
\usepackage{hyperref}
\usepackage{fourier}
\usepackage{tabularx}
\usepackage{amssymb}
\usepackage{chemformula}
\usepackage{appendix}
\usepackage{multirow}
\usepackage{graphicx}
\usepackage{makecell}
\usepackage{adjustbox}
\usepackage{listings}
\usepackage{xcolor}
\usepackage{ulem}
\usepackage{enumitem}

% =====================================================================================
% Paper Define
% =====================================================================================

\hypersetup{
    colorlinks,
    linktoc=all,
    citecolor=black,
    filecolor=black,
    linkcolor=black,
    urlcolor=cyan
}

\geometry{a4paper,left=2cm,right=2cm,top=3cm,bottom=3cm}

\CJKfamily{\fangsong}

% 行内高亮
\lstdefinestyle{inlinecode}{
    basicstyle=\ttfamily\small,
    backgroundcolor=\colorbox{gray!10},
    breaklines=true,
    frame=none
}

\newcommand{\code}[1]{\colorbox{gray!10}{\lstinline[style=inlinecode]|#1|}}

% 行间高亮
\lstdefinestyle{codeblock}{
    basicstyle=\ttfamily\small,
    backgroundcolor=\color{gray!10},
    frame=single,
    frameround=tttt,
    framerule=0.5pt,
    rulecolor=\color{gray!50},
    breaklines=true,
    breakatwhitespace=false,
    numbers=left,
    numberstyle=\tiny\color{gray},
    stepnumber=1,
    numbersep=8pt,
    showspaces=false,
    showstringspaces=false,
    showtabs=false,
    tabsize=4,
    xleftmargin=15pt,
    xrightmargin=5pt,
    aboveskip=10pt,
    belowskip=10pt
}

\lstset{style=codeblock}

% 定义命令:双下划线 或者 波浪线
% \newcommand{\integrated}[1]{\uuline{#1}} % 双下划线
\newcommand{\integrated}[1]{\uwave{#1}} % 波浪线

% =====================================================================================
% Begin Document
% =====================================================================================

\begin{document}

% =====================================================================================
% Custom Title Page
% =====================================================================================
\pagestyle{empty}

% 自定义标题页
\begin{center}
    \vspace*{2cm}
    {\Huge \textbf{HGD(RK3588) 系统指南}}
    \vspace{2cm}
\end{center}

\section*{文档版本}

\begin{center}
    \begin{tabular}{|*{5}{c|}} \hline
        编写日期 & 版本编号 & 编写 & 校对 & 修改说明 \\ \hline
        2025-12-12 & 1.0 & 肖劲涛  & 初始版本 & -- \\ \hline
    \end{tabular}
\end{center}

\vspace{1cm}

\section*{软件版本}

\begin{center}
    \begin{tabular}{|*{5}{c|}} \hline
        类型 & 发布日期 & Commit ID & 发布人员 & Tag|Branch \\ \hline
        整体系统 & 2025-12-15 & 无 & 肖劲涛 & HGD\_1  \\ \hline
        Kernel & 2025-12-11 & 614e185ad3aaf9866a68c51e5979be8cab8a58a4 & 肖劲涛  & HGD\_NewLogo \\ \hline
        Buildroot & 2025-5-30 & 1c124ea63c7e2033443269fd9d2a7da1aa264181 & 肖劲涛  & master \\ \hline
        Device & 2025-5-30 & d66165e16232f7f38a000fd364e8c31477e279ab & 肖劲涛 & master \\ \hline
        image\_parts & 2025-10-3 & ac04f50c8ec47486860c3ed965c28ae9f432b85d & 肖劲涛 & master \\ \hline
        rkbin & 2025-2-22 & a2a0b89b6c8c612dca5ed9ed8a68db8a07f68bc0 & 肖劲涛 & master \\ \hline
    \end{tabular}
\end{center}

\vspace{\fill}

\begin{center}
    {\large \today}
\end{center}

\thispagestyle{empty}

% =====================================================================================
% Header Page 正文前内容
% =====================================================================================
\newpage
\pagestyle{plain}
\pagenumbering{Roman}

\pagestyle{fancy}

\setlength{\headsep}{40pt}
\setlength{\footskip}{40pt}

% 页眉横线宽度
\renewcommand\headrulewidth{.5pt}
% 清空页眉
\fancyhead{}
% 设置居中的页眉内容
\fancyhead[C]{HGD(RK3588) 系统指南}

\renewcommand\footrulewidth{.5pt}
\fancyfoot{}
\fancyfoot[C]{\thepage}

% =====================================================================================
% Version
% =====================================================================================

\section*{仓库与位置}

上游组位置位于\href{http://10.42.0.252:12123/general/sdk/rockchip}{General/SDK/rockchip},几个子仓库分别内容如下:

\begin{enumerate}
    \item \text{RK3588\_Buildroot}对应\code{/buildroot}
    \item \text{RK3588\_Device}对应\code{/device}
    \item \text{RK3588\_ImageParts}对应\code{/image\_parts}
    \item \text{RK3588\_Kernel}对应\code{/kernel}
    \item \text{RK3588\_rkbin}对应\code{/rkbin}
\end{enumerate}

\noindent 其中\text{RK3588\_ImageParts}打包了包括如下的分区:

\begin{enumerate}
    \item \code{app},应用程序
    \item \code{oemven},可替换\text{OEM}资源
    \item \code{vendor},制造商分区,用于保留\text{lib}库等
    \item \code{hold},不更变分区,用于保留永久数据等
    \item \code{userdata},用户数据分区
\end{enumerate}

\section*{文档结构}

第一章将讲述各类系统模块如何调用,适用于项目系统发布人员、应用开发人员。

第二章将讲述\text{Buildroot/Fs-Overlay}中的内容,适用于项目系统发布人员、应用开发人员。该章节中将首要地介绍完整的系统启动流程,以及各模块的使用逻辑;之后将介绍\text{Buildroot/FS-Overlay}中的文件结构,以及针对不同的项目应该如何进行修改。

第三章将讲述\text{EMMC}分区设计,适用于系统发布人员。

\newpage

% =====================================================================================
% TABLE OF CONTENTS
% =====================================================================================
\renewcommand{\contentsname}{目录}
\tableofcontents
\newpage

% =====================================================================================
% Main Body 正文内容
% =====================================================================================
\pagenumbering{arabic}

% =====================================================================================
% Section 1 : 系统模块
% =====================================================================================

\section{系统模块}

% ----------------------------------------
%  Subsection 1.1 : GPIO
% ----------------------------------------

\subsection{GPIO}

\text{HGD}将所有需要被调用的\text{GPIO}管脚均封装为了\code{/dev/jp\_hgd\_gpio\_ctl\_enable\_\*}设备,列表如下:

\begin{lstlisting}[language=sh]
root@HGD:/# ls /dev/jp_hgd_gpio_ctl_enable_*
/dev/jp_hgd_gpio_ctl_enable_evf
/dev/jp_hgd_gpio_ctl_enable_panel
/dev/jp_hgd_gpio_ctl_enable_laser
/dev/jp_hgd_gpio_ctl_enable_refrigerator
\end{lstlisting}

对于上述“设备”,均可以使用\code{echo 1}或者\code{echo on}来进行使能,例如:

\begin{lstlisting}[language=sh]
root@HGD:/# echo 1 > /dev/jp_hgd_gpio_ctl_enable_evf
\end{lstlisting}

% ----------------------------------------
%  Subsection 1.2 : UART
% ----------------------------------------

\subsection{UART}

当前的系统中主要提供如下的串口:

\begin{enumerate}
    \item \code{/dev/ttyS1},电源\text{MCU}
    \item \code{/dev/ttyS2},调试串口
    \item \code{/dev/ttyS4},对焦\text{MCU}
    \item \code{/dev/ttyS7},连接\text{GPS}
\end{enumerate}

系统提供了一个\code{uart_listen}的串口监听程序,以对焦\text{MCU}通讯为例:

\begin{lstlisting}[language=sh]
root@HGD:/root/app/utils# ./uart_listen 115200 /dev/ttyS4 &
root@HGD:/root/app/utils# echo -en '\x43\x44\x4A\x50\x02\x02\x01\x00\x10\x0C' > /dev/ttyS4
root@HGD:/root/app/utils# Received data: 43 44 4A 50 02 02 01 00 10 FE F2 
\end{lstlisting}
    
% ----------------------------------------
%  Subsection 1.3 : I2C
% ----------------------------------------

\subsection{I2C}

当前的系统中\text{I2C}配置情况如下:

\begin{enumerate}
    \item \code{i2c2}
    \begin{itemize}
        \item \integrated{触摸屏:\text{GT9xx}}
    \end{itemize}

    \item \code{i2c3}
    \begin{itemize}
        \item 地磁传感器:\text{MMC5603NJ}
    \end{itemize}

    \item \code{i2c3}
    \begin{itemize}
        \item \integrated{可见光摄像头:\text{IMX335}}
    \end{itemize}

    \item \code{i2c6}
    \begin{itemize}
        \item \integrated{\text{RTC}时钟:\text{PCF8536T}}
        \item 温度传感器:\text{MTS4P}
    \end{itemize}

    \item \code{i2c7}
    \begin{itemize}
        \item 陀螺仪:\text{ICM-20600}
        \item \integrated{音频芯片:\text{ES8311}}
    \end{itemize}
\end{enumerate}

\noindent 其中用波浪线标识的设备均已集成在系统中。

板端提供了如下的工具用于查询并读写\text{I2C}设备:

\begin{lstlisting}[language=sh]
root@HGD:/# i2cdetect -l 
i2c-0	i2c       	rk3x-i2c                        	I2C adapter
i2c-1	i2c       	rk3x-i2c                        	I2C adapter
i2c-2	i2c       	rk3x-i2c                        	I2C adapter
i2c-3	i2c       	rk3x-i2c                        	I2C adapter
i2c-4	i2c       	rk3x-i2c                        	I2C adapter
i2c-6	i2c       	rk3x-i2c                        	I2C adapter
i2c-7	i2c       	rk3x-i2c                        	I2C adapter
i2c-9	i2c       	ddc                             	I2C adapter
root@HGD:/# i2cdetect -y 7
        0  1  2  3  4  5  6  7  8  9  a  b  c  d  e  f
00:                         -- -- -- -- -- -- -- -- 
10: -- -- -- -- -- -- -- -- -- UU -- -- -- -- -- -- 
20: -- -- -- -- -- -- -- -- -- -- -- -- -- -- -- -- 
30: -- -- -- -- -- -- -- -- -- -- -- -- -- -- -- -- 
40: -- -- -- -- -- -- -- -- -- -- -- -- -- -- -- -- 
50: -- -- -- -- -- -- -- -- -- -- -- -- -- -- -- -- 
60: -- -- -- -- -- -- -- -- -- 69 -- -- -- -- -- -- 
70: -- -- -- -- -- -- -- --                         
root@HGD:/# i2ctransfer -y 7 w2@0x69 0x00 0x0c r1
\end{lstlisting}

% ----------------------------------------
%  Subsection 1.4 : 按键
% ----------------------------------------

\subsection{Keys}

系统一共提供了三组按键,其中两组为\text{ADC Key}、一组为\text{GPIO Key},以\code{evtest}为例:

\begin{lstlisting}[language=sh]
root@HGD:/# evtest 
No device specified, trying to scan all of /dev/input/event*
Available devices:
/dev/input/event0:	rk805 pwrkey
/dev/input/event1:	goodix-ts
/dev/input/event2:	adc-keys-ch1
/dev/input/event3:	adc-keys-ch3
/dev/input/event4:	rockchip-hdmi0 rockchip-hdmi0
/dev/input/event5:	gpio-keys
\end{lstlisting}

\noindent 其中,按键定义如下:

\begin{enumerate}
    \item \code{/dev/input/event2} \text{adc-keys-ch1}
        \begin{itemize}
            \item Event type 0 (EV\_SYN)
            \item Event type 1 (EV\_KEY)
                \begin{enumerate}[label=\arabic*.]
                    \item Event code 2 (KEY\_1)
                    \item Event code 3 (KEY\_2)
                    \item Event code 4 (KEY\_3)
                    \item Event code 5 (KEY\_4)
                    \item Event code 6 (KEY\_5)
                \end{enumerate}
        \end{itemize}

    \item \code{/dev/input/event3} \text{adc-keys-ch3}
        \begin{itemize}
            \item Event type 0 (EV\_SYN)
            \item Event type 1 (EV\_KEY)
                \begin{enumerate}[label=\arabic*.]
                    \item Event code 59 (KEY\_F1)
                    \item Event code 60 (KEY\_F2)
                    \item Event code 61 (KEY\_F3)
                    \item Event code 62 (KEY\_F4)
                    \item Event code 63 (KEY\_F5)
                    \item Event code 64 (KEY\_F6)
                \end{enumerate}
        \end{itemize}

    \item \code{/dev/input/event5} \text{gpio-keys}
        \begin{itemize}
            \item Event type 0 (EV\_SYN)
            \item Event type 1 (EV\_KEY)
                \begin{enumerate}[label=\arabic*.]
                    \item Event code 30 (KEY\_A)
                    \item Event code 31 (KEY\_S)
                    \item Event code 46 (KEY\_C)
                \end{enumerate}
        \end{itemize}
\end{enumerate}

% ----------------------------------------
%  Subsection 1.5 : USB
% ----------------------------------------

\subsection{USB}

系统在如下的位置提供了\text{USB}使能的脚本:

\begin{lstlisting}[language=sh]
root@HGD:/root/sys/usb# ls
adb.sh  mtp.sh    
\end{lstlisting}

\noindent 其中,\code{adb.sh}开启\text{ADB}和\text{RNDIS}功能;\code{mtp}开启\text{MTP}服务。\text{MTP}服务由\code{mpt-server}实现,如有自定义需求,可以自行修改\text{SDK-Buildroot}中的相关配置。

% ----------------------------------------
%  Subsection 1.6 : WiFi
% ----------------------------------------

\subsection{WiFi}

系统在如下的位置提供了\text{WiFi}使能的脚本:

\begin{lstlisting}[language=sh]
root@HGD:/root/sys/wifi# ./wifi.sh --help
Usage: ./wifi.sh [ACTION] [SSID] [PASSWORD]
Actions:
  insmod            - Load WiFi module
  scan              - Scan available WiFi networks
  connect SSID PWD  - Connect to WiFi (first time)
  reconnect SSID PWD- Reconnect to different WiFi
  ap SSID PWD       - Start WiFi Access Point mode (PWD: 8-63 chars)
Examples:
  ./wifi.sh insmod
  ./wifi.sh scan
  ./wifi.sh connect MyWiFi MyPassword
  ./wifi.sh reconnect NewWiFi NewPassword
  ./wifi.sh ap MyHotspot MyPassword123
\end{lstlisting}

\noindent 其中,\text{WiFi}的核心配置部分在\code{/etc/hostapd}中实现,其部分参考配置如下:

\begin{lstlisting}[language=sh]
interface=${INTERFACE}
driver=nl80211
ssid=${WIFISSID}
hw_mode=g
channel=7
wmm_enabled=0
macaddr_acl=0
auth_algs=1
ignore_broadcast_ssid=0
wpa=2
wpa_passphrase=${WIFIPWD}
wpa_key_mgmt=WPA-PSK
wpa_pairwise=TKIP
rsn_pairwise=CCMP
\end{lstlisting}

同时,\code{scan}的结果将存放在\code{/var/run/wpa_supplicant/scan.txt}中,内容以如下的格式存放:

\begin{lstlisting}[language=sh]
# Scan Results (2025-05-15 07:18:50)
# BSSID / Frequency / Signal Level / Flags / SSID
2a:d0:f5:2b:ca:2b 2422 -64 [WPA2-PSK-CCMP][ESS] AHGZ-2.4
be:fc:e7:5f:a9:e2 5260 -72 [WPA2-PSK-CCMP][WPS][ESS] ESW
c4:70:ab:7a:6a:bb 5180 -65 [WPA-PSK-CCMP][WPA2-PSK-CCMP][ESS] AHGZ
c6:70:ab:1a:6a:bb 5180 -65 [WPA2-PSK-CCMP][ESS] AHGZ-5G
bc:fc:e7:5f:a9:e2 2442 -55 [WPA2-PSK-CCMP][WPS][ESS] ESW
d8:15:0d:f0:b6:b8 2412 -70 [WPA-PSK-CCMP][WPA2-PSK-CCMP][WPS][ESS] xwc802
c6:70:ab:aa:6a:bb 2467 -46 [WPA2-PSK-CCMP][ESS] AHGZ-2.4
cc:4b:73:f5:1c:88 2442 -65 [WPA2-PSK-CCMP][ESS] NightEyes-76
d6:31:27:83:b9:f6 5785 -79 [WPA-PSK-CCMP][WPA2-PSK-CCMP][ESS] AHGZ
d6:31:27:93:b9:f6 5785 -79 [WPA2-PSK-CCMP][ESS] AHGZ-5G
10:2c:6b:08:f1:5a 2457 -72 [WPA2-PSK-CCMP][ESS] NightEyes-150
f8:cd:c8:54:43:ec 2417 -77 [WPA-PSK-CCMP+TKIP][WPA2-PSK-CCMP+TKIP][WPS][ESS] ChinaNet-Urca
b0:44:14:ef:ec:e0 5745 -89 [WPA2-PSK-CCMP][ESS] CDCL
b0:44:14:ef:85:00 5260 -89 [WPA2-PSK-CCMP][ESS] CDCL
94:d9:b3:24:11:9a 2437 -84 [WPA-PSK-CCMP+TKIP][WPA2-PSK-CCMP+TKIP][ESS] Tedu_class_2.4
\end{lstlisting}

值得注意的是,由于制板工艺问题,可能存在\text{WiFi}模块加载模块失败的情况,此时可以尝试重启,确保在开机阶段有如下的输出:

\begin{lstlisting}[language=sh]
[   12.401401] [dhd] ======== PULL WL_REG_ON(-1) LOW! ========
[   12.401419] [WLAN_RFKILL]: jiang rockchip_wifi_power: 0
[   12.401436] [BT_RFKILL]: rfkill_get_bt_power_state: rfkill-bt driver has not Successful initialized
[   12.401453] [BT_RFKILL]: rfkill_get_bt_power_state: rfkill-bt driver has not Successful initialized
[   12.401470] [WLAN_RFKILL]: wifi shut off power [GPIO-1-1]
[   12.401487] [dhd] [wlan0] wl_android_wifi_off : out
[   12.401519] [dhd] dhd_os_wake_lock_destroy: wake lock count=1
[   12.401673] [dhd] [wlan0] dhd_stop : Exit
[   12.401707] [dhd] [wlan0] dhd_open : Exit ret=-1
[   12.401726] [dhd] Failed to open primary dev ret -1
[   12.401742] [dhd] dhd_pri_open : mutex is released.
\end{lstlisting}

% ----------------------------------------
%  Subsection 1.7 : SSH & sFTP
% ----------------------------------------

\subsection{SSH and sFTP}

系统默认开启\text{SSH}和\text{sFTP}服务,可以通过如此的方式查看设备\text{IP}地址:

\begin{lstlisting}[language=sh]
root@HGD:/# ifconfig 
lo        Link encap:Local Loopback  
          inet addr:127.0.0.1  Mask:255.0.0.0
          inet6 addr: ::1/128 Scope:Host
          UP LOOPBACK RUNNING  MTU:65536  Metric:1
          RX packets:266 errors:0 dropped:0 overruns:0 frame:0
          TX packets:266 errors:0 dropped:0 overruns:0 carrier:0
          collisions:0 txqueuelen:1000 
          RX bytes:19197 (18.7 KiB)  TX bytes:19197 (18.7 KiB)

usb0      Link encap:Ethernet  HWaddr 96:2D:C9:F6:EC:1F  
          inet addr:169.254.43.1  Bcast:169.254.255.255  Mask:255.255.0.0
          inet6 addr: fe80::942d:c9ff:fef6:ec1f/64 Scope:Link
          UP BROADCAST RUNNING MULTICAST  MTU:1500  Metric:1
          RX packets:187 errors:0 dropped:0 overruns:0 frame:0
          TX packets:12 errors:0 dropped:0 overruns:0 carrier:0
          collisions:0 txqueuelen:1000 
          RX bytes:31825 (31.0 KiB)  TX bytes:1464 (1.4 KiB)

wlan0     Link encap:Ethernet  HWaddr CC:64:1A:75:46:A0  
          inet addr:192.168.50.96  Bcast:192.168.50.255  Mask:255.255.255.0
          UP BROADCAST RUNNING MULTICAST  MTU:1500  Metric:1
          RX packets:105 errors:0 dropped:0 overruns:0 frame:0
          TX packets:88 errors:0 dropped:0 overruns:0 carrier:0
          collisions:0 txqueuelen:1000 
          RX bytes:10705 (10.4 KiB)  TX bytes:7404 (7.2 KiB)
\end{lstlisting}

\noindent 其中,若要调整\code{usb0}的地址,请自行根据\text{RFC1918}或\text{RFC3927}进行相关内容配置;\code{wlan0}支持\text{AP}模式,其地址将在配置后显示:

\begin{lstlisting}[language=sh]
Configuring dnsmasq DHCP server...
Starting dnsmasq DHCP server...
dnsmasq DHCP server started successfully
WiFi AP mode started successfully!
AP SSID: HaHa
AP IP: 192.168.4.1
DHCP Range: 192.168.4.20 - 192.168.4.200
DNS Servers: 8.8.8.8, 8.8.4.4
Access Point started successfully
\end{lstlisting}

当前的登录凭证为:用户名\code{root},密码\code{1}。

% =====================================================================================
% End Document
% =====================================================================================

\end{document}