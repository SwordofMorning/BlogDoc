\documentclass[UTF8]{ctexart}

\usepackage{ctex}
\usepackage{times}
\usepackage{geometry}
\usepackage{subcaption}
\usepackage{textcomp}
\usepackage{tocbibind}
\usepackage{fancyhdr}
\usepackage{color}
\usepackage{graphicx}
\usepackage{titling}
\usepackage{hyperref}
\usepackage{fourier}
\usepackage{tabularx}
\usepackage{amssymb}
\usepackage{chemformula}
\usepackage{appendix}
\usepackage{multirow}
\usepackage{graphicx}
\usepackage{makecell}
\usepackage{adjustbox}
\usepackage{listings}
\usepackage{xcolor}

% -------------------------------------------------------------------------------------
% Paper Define
% -------------------------------------------------------------------------------------

\hypersetup{
    colorlinks,
    linktoc=all,
    citecolor=black,
    filecolor=black,
    linkcolor=black,
    urlcolor=cyan
}

\geometry{a4paper,left=2cm,right=2cm,top=3cm,bottom=3cm}

\CJKfamily{\fangsong}

% 行内高亮
\lstdefinestyle{inlinecode}{
    basicstyle=\ttfamily\small,
    backgroundcolor=\colorbox{gray!10},
    breaklines=true,
    frame=none
}

\newcommand{\code}[1]{\colorbox{gray!10}{\lstinline[style=inlinecode]|#1|}}

% 行间高亮
\lstdefinestyle{codeblock}{
    basicstyle=\ttfamily\small,
    backgroundcolor=\color{gray!10},
    frame=single,
    frameround=tttt,
    framerule=0.5pt,
    rulecolor=\color{gray!50},
    breaklines=true,
    breakatwhitespace=false,
    numbers=left,
    numberstyle=\tiny\color{gray},
    stepnumber=1,
    numbersep=8pt,
    showspaces=false,
    showstringspaces=false,
    showtabs=false,
    tabsize=4,
    xleftmargin=15pt,
    xrightmargin=5pt,
    aboveskip=10pt,
    belowskip=10pt
}

\lstset{style=codeblock}

% -------------------------------------------------------------------------------------
% Begin Document
% -------------------------------------------------------------------------------------

\begin{document}

% -------------------------------------------------------------------------------------
% Custom Title Page
% -------------------------------------------------------------------------------------
\pagestyle{empty}

% 自定义标题页
\begin{center}
    \vspace*{2cm}
    {\Huge \textbf{VS8x9 系统指南}}
    \vspace{2cm}
\end{center}

\section*{文档版本}

\begin{center}
    \begin{tabular}{|*{5}{c|}} \hline
        编写日期 & 版本编号 & 编写 & 校对 & 修改说明 \\ \hline
        2025-10-13 & 1.2 & 肖劲涛  & 适配第\text{9}版本系统 & -- \\ \hline
        2025-09-23 & 1.1 & 肖劲涛  & 增加\text{GPIO}和\text{UART}说明 & -- \\ \hline
        2025-09-18 & 1.0 & 肖劲涛  & 初始版本 & -- \\ \hline
    \end{tabular}
\end{center}

\vspace{1cm}

\section*{软件版本}

\begin{center}
    \begin{tabular}{|*{5}{c|}} \hline
        类型 & 发布日期 & Commit ID & 发布人员 & Tag|Branch \\ \hline
        整体系统 & 2025-10-09 & -- & 肖劲涛 & m2\_ver9  \\ \hline
        mk & 2025-08-25 & 73be3c85d16ab0c60c647303aa5853a721ae2c2b & 肖劲涛  & master \\ \hline
        vs-tools & 2025-09-19 & 37349a26e6a87245e3e504164d660cdebcc69f48 & 肖劲涛  & master \\ \hline
        U-Boot & 2025-09-09 & 493afc5a6860fc1224dfa64530bc4b17f7545e52 & 肖劲涛  & master \\ \hline
        Kernel & 2025-09-25 & a354ea130855fc54b51b607ae7ec875081731f47 & 肖劲涛  & master \\ \hline
        Fs-Overlay & 2025-10-09 & d4e82cd4574b936692503bd1b1de8936da93004a & 肖劲涛  & master \\ \hline
    \end{tabular}
\end{center}

\vspace{\fill}

\begin{center}
    {\large \today}
\end{center}

\thispagestyle{empty}

% -------------------------------------------------------------------------------------
% Header Page 正文前内容
% -------------------------------------------------------------------------------------
\newpage
\pagestyle{plain}
\pagenumbering{Roman}

\pagestyle{fancy}

\setlength{\headsep}{40pt}
\setlength{\footskip}{40pt}

% 页眉横线宽度
\renewcommand\headrulewidth{.5pt}
% 清空页眉
\fancyhead{}
% 设置居中的页眉内容
\fancyhead[C]{VS8x9 系统指南}

\renewcommand\footrulewidth{.5pt}
\fancyfoot{}
\fancyfoot[C]{\thepage}

% -------------------------------------------------------------------------------------
% Version
% -------------------------------------------------------------------------------------

\section*{仓库与位置}

上游组位置位于\href{http://192.168.110.252:12123/general/sdk/visinex}{General/SDK/Visinex},几个子仓库分别内容如下:

\begin{enumerate}
    \item \text{mk}对应\code{customer-rel/board/package/mk}
    \item \text{vs-tools}对应\code{customer-rel/board/package/vs-tools}
    \item \text{U-Boot}对应\code{customer-rel/board/package/opensource/u-boot}
    \item \text{Kernel}对应\code{customer-rel/board/package/opensource/kernel/linux}
    \item \text{Fs-Overlay}对应\code{customer-rel/board/package/fs-overlay}
\end{enumerate}

\noindent 其中\text{Fs-Overlay}为\text{busybox}镜像的增量替换,具体内容将在后续章节阐述。

\section*{文档结构}

第一章将讲述基本的\text{SDK}编译、烧录方法,适用于系统发布人员。

第二章将讲述\text{FS-Overlay}中的内容,适用于项目系统发布人员、应用开发人员。该章节中将首要地介绍完整的系统启动流程,以及各模块的使用逻辑;之后将介绍\text{FS-Overlay}中的文件结构,以及针对不同的项目应该如何进行修改。

第三章将讲述\text{EMMC}分区设计,适用于系统发布人员。其中包含\text{vs-tools}中的\code{xml_gen.sh}、\text{U-Boot}中的自定义内容,以及应用层的\text{OTA}设计。如应用开发人员需要修改相应的\text{OTA}功能,以支持更灵活的升级方式,则应当完整地了解本章的内容。

\newpage

% -------------------------------------------------------------------------------------
% TABLE OF CONTENTS
% -------------------------------------------------------------------------------------
\renewcommand{\contentsname}{目录}
\tableofcontents
\newpage

% -------------------------------------------------------------------------------------
% Main Body 正文内容
% -------------------------------------------------------------------------------------
\pagenumbering{arabic}

% -------------------------------------------------------------------------------------
% Section 1 : 基础编译
% -------------------------------------------------------------------------------------

\section{基础编译}

\subsection{预准备}

\begin{enumerate}
    \item 在\text{Windows}上解压\code{VS839_M1_release.rar}
    \item 将\code{vssdk_rel_CS_R1.2.5.tgz}和\code{sdk_unpack.sh}拷贝到\text{Ubuntu}
    \item 执行\code{./sdk_unpack.sh}解压\text{SDK}
    \item 修改\code{customer-rel/board/package/mk/common.mk}中的编译器路径
    \item 拷贝内核\code{linux-5.4.94.tar.gz}到\code{customer-rel/board/package/opensource/kernel}并解压
    \item 重命名\code{linux-5.4.94}为\code{linux}并打补丁,参考如下的代码块
\end{enumerate}

\begin{lstlisting}[language=sh]
mv linux-5.4.94 linux
cd linux
patch -p1 < ../linux-5.4.94.patch
\end{lstlisting}

在正式编译系统之前,请先通过之前的\text{git}仓库同步最新的源码。根据不同的项目,请确保同步至不同的下游仓库,而不是直接同步上游仓库。

\subsection{Makefile}

\subsubsection{Fs-Overlay修改}

如果你同步了上述仓库,并且需要使用\text{Fs-Overlay}来为\text{busybox}镜像增量打包,那么需要首要地修改如下的\text{Makefile}文件:

\begin{lstlisting}
customer-rel/board/package/Makefile
\end{lstlisting}

在其中第\text{245}行,\text{Making rootfs image}之前添加如下内容

\begin{lstlisting}[language=sh]
@echo "Applying fs-overlay..."
@if [[ -d $(FS_OVERLAY_PATH) ]]; then \
    echo "Copying fs-overlay files to rootfs..."; \
    $(CP) $(FS_OVERLAY_PATH)/* $(STRIP_ROOTFS_PATH)/ -rfap; \
    echo "fs-overlay applied successfully."; \
else \
    echo "No fs-overlay directory found, skipping..."; \
fi
\end{lstlisting}

\subsubsection{DSP修改}

\text{DSP}部分由原厂预编译完成,\text{SDK}中无相关代码,因此需要注释相关选线,参考如下的配置:

\begin{lstlisting}[language=sh]
PACKAGES = \
	atf \
	uboot \
	busybox \
	vs-osal \
	vs-ispsensor \
	vs-sample \
	vs-sign \
	ros
\end{lstlisting}

\subsubsection{DDR修改}

修改\text{Makefile}的\code{BOARD_TYPE}以适配\text{LPDDR4}的主板,内容如下:

\begin{lstlisting}[language=sh]
BOARD_TYPE ?= ci03
\end{lstlisting}

\noindent 修改后的完整\text{Makefile}请参考\href{http://192.168.110.252:12123/general/sdk/visinex/vs8x9_fs-overlay/-/issues/1}{Makefile For Fs-Overlay
}议题。

\subsection{编译}

在完成了上述修改之后,切换到\code{customer-rel/board/package}下,通过执行命令\code{make all}来进行第一次的整体编译。

\subsection{输出路径}

\text{Kernel}和\text{FS}的输出路径为:

\begin{lstlisting}[language=sh]
customer-rel/board/package/image/linux_ext4
\end{lstlisting}

\text{U-Boot}以及板端\text{xml}配置文件输出路径为:

\begin{lstlisting}[language=sh]
customer-rel/board/package/image/boot/ci03/vs8x9/lpddr4/single_rank_3200
\end{lstlisting}

\noindent 其中\code{ci03}指\text{Makefile}中的\text{DDR}选项,与前述内容相同;\code{lpddr4}与\code{single_rank_3200}指板端\text{DDR}的具体配置,需要在\code{vs-tools}修改相关配置实现。

\subsection{烧录方法}

\subsubsection{启动镜像烧写}

如要通过串口单独烧写启动镜像(\text{bootloader})的部分内容,操作流程如下:

\begin{enumerate}
    \item 在\text{Windows}上打开\text{VS Burn Tool}工具
    \item 传输方式选择串口,同时确保芯片此时可通过串口烧录
    \item 在\textbf{启动镜像烧写}一栏中选择默认的\text{xml}文件
    \item 将\text{Flash}类型修改为\text{EMMC}
    \item 确保\text{Package}与\text{xml}中的描述符相一致
    \item 点击\textbf{烧写}按钮,等待完成即可
\end{enumerate}

\subsubsection{Linux镜像烧写}

由于板端设计原因,因此本章节只讲述\text{USB}烧录方法。该模式适用于启动镜像烧写之后的整体镜像烧写流程,也适用于重新烧写已有系统的部分或整体分区。

\begin{enumerate}
    \item 通过\text{Debug}串口进入\text{U-Boot}的命令行模式
    \item 输入\code{usb device}命令,同时\text{Windows}端通过\text{USB}线连接到板端\text{USB1}接口
    \item 传输方式选择\text{USB}口
    \item 在\textbf{Linux镜像烧写}一栏中选择默认的\text{xml}文件
    \item 将\text{Flash}类型修改为\text{EMMC}
    \item 选中表格中需要烧写的分区
    \item 点击\textbf{烧写}按钮,等待完成即可
\end{enumerate}

\newpage

% -------------------------------------------------------------------------------------
% Section 2 : 系统模块
% -------------------------------------------------------------------------------------

\section{系统模块}

\subsection{开机启动}

系统启动时,默认调用\code{/etc/init.d/rcS}脚本。

在\text{Fs-Overlay}中,对其新增了一个引导至\code{/root/init/rcS}的内容:

\begin{lstlisting}[language=sh]
# sleep for stabilizing
echo "Run rcS, countdown 2..."
sleep 1
echo "Run rcS, countdown 1..."
sleep 1
/root/init/rcS
\end{lstlisting}

\noindent 其中的\text{sleep}时间可以根据自行实际情况修改或取消。

在\code{/root/init/rcS}中,定义了如下的脚本启动流程:

\begin{itemize}
    \item 读取当前目录下的所有符合\code{s??*}的脚本或可执行程序
    \item 按照字典顺序将其排序
    \item 按照从小到大的顺序依次执行上述脚本或程序
    \item 当某个脚本执行失败后,继续执行下一个脚本
\end{itemize}

\noindent 其中提供了如下的内容:

\begin{lstlisting}[language=sh]
# Optional: Stop the entire initialization process if a critical script fails
# Uncomment the following to enable strict mode
# echo "Critical script failed, stopping initialization"
# exit $RESULT
\end{lstlisting}

\noindent 取消注释则让\code{rcS}在某一个脚本执行失败之后退出,而不继续执行剩下的脚本。

\subsection{初始化脚本}

\subsubsection{insmod}

\code{s10_insmod_pre.sh}和\code{s20_insmod_post.sh}一起实现了\code{load8x9.sh}中的加载模块的功能;区别在于,前者只加载和屏幕相关的部分内容,后者加载其余的模块。

这两个脚本应当由系统调用,用户层空间不做任何操作。

\subsubsection{bootanimation}

\code{s15_bootanimation.sh}提供了一个开机动画脚本:

\begin{lstlisting}[language=sh]
start)
    /usr/bin/bootanimation /root/oem/logo.bmp &
    ;;
stop)
    killall /usr/bin/bootanimation
    ;;
\end{lstlisting}

\noindent 其中,\code{/root/oem/logo.bmp}可以替换为任意分辨率为$1280 \times 960$的\text{bmp}图片。

\subsubsection{ota}

\code{s30_ota.sh}实现\text{OTA}分区检测功能。该脚本由系统自行维护,用户层不应该执行任何对这个脚本的操作。

\subsubsection{mount}

\code{s40_mount.sh}提供了挂载磁盘分区的选项,同时包含格式化等功能。

该脚本提供如下的选项:

\begin{itemize}
    \item \code{start}挂载\code{vendor}、\code{app}以及\code{userdata}分区,如果该分区未格式化则先格式化再挂载
    \item \code{stop}卸载\code{vendor}、\code{app}以及\code{userdata}分区
    \item \code{restart}卸载并重新挂载上述分区
    \item \code{status}显示当前的挂在信息和文件系统类型
    \item \code{format <name>}以\code{ext2}格式化所有分区,其中参数\text{name}可以用来指定某个单独分区。
\end{itemize}

\subsubsection{sysconf}

\code{s50_sysconf.sh}提供了系统配置,主要实现\text{pinctrl}、低功耗等配置。该脚本由系统自行维护,用户层不应该执行任何对这个脚本的操作。

\subsubsection{usb}

\code{s80_usb.sh}提供了包含\text{RNDIS}、\text{Serial}以及\text{MTP}的相关配置功能。

该脚本提供如下的选项:

\begin{itemize}
    \item \code{start}默认引导至\text{RNDIS}模式
    \item \code{rndis}切换\text{USB}至\text{RNDIS}模式
    \item \code{serial}切换\text{USB}至\text{Serial}模式
    \item \code{mtp}切换\text{USB}至\text{MTP}模式
    \item \code{stop}停止当前的所有\text{USB}功能
    \item \code{restart}重启\text{USB}功能至\text{RNDIS}模式
    \item \code{status}显示当前的\text{USB}状态
    \item \code{config}显示当前的\text{USB}配置信息
\end{itemize}

同时在\code{/root/init}下,提供了一个\code{usb.conf}的文件,该文件中包含了如下的\text{USB}配置信息:

\begin{itemize}
    \item \text{RNDIS}模式的\text{IP}地址、掩码和网关
    \item \text{Serial}模式下的\text{USB}设备信息
    \item \text{MTP}模式下的\text{USB}设备信息
    \item \text{MTP}模式下的映射路径与设备名
    \item 其他通用配置
\end{itemize}

\subsubsection{deb}

\code{s90_deb.sh}将在开机后检测应用程序是否安装,如果没有安装,则将自动为其安装\code{/root/deb/}下的应用。该脚本由系统自行维护,用户层不应该执行任何对这个脚本的操作。

\subsubsection{app}

\code{app}

\code{s95_app.sh}将跳转至\code{/app/rcS},执行应用程序启动。

\subsection{功能模块}

\subsubsection{串口}

板端留有\text{Debug}串口,串口设置如下:

\begin{enumerate}
    \item 波特率:\text{115200}
    \item 数据位:\text{8}
    \item 停止位:\text{1}
    \item 无奇偶校验、无控制流
\end{enumerate}

通过串口连接至板端无需用户名与密码。

\subsubsection{USB}

当前的系统支持\text{RNDIS}、\text{Serial}以及\text{MTP}三种模式,目前只能单独使用其中一种模式。在用户空间,可以通过命令行的方式调用\code{/root/init/s50_usb.sh}来切换不同的状态。

切换至\text{RNDIS}模式:

\begin{lstlisting}[language=sh]
~ # /root/init/s50_usb.sh rndis
Loading configuration from /root/init/usb.conf
Configuration loaded successfully
Stopping all USB services...
Unloading module: g_ether
Unloading module: usb_f_ecm
Unloading module: usb_f_rndis
Unloading module: u_ether
Unloading module: libcomposite
All USB services stopped
Starting USB RNDIS (Network) mode...
Loading module: libcomposite
Module libcomposite loaded successfully
Loading module: u_ether
Module u_ether loaded successfully
Loading module: usb_f_rndis
Module usb_f_rndis loaded successfully
Loading module: usb_f_ecm
Module usb_f_ecm loaded successfully
Loading module: g_ether
Module g_ether loaded successfully
All RNDIS modules loaded successfully
RNDIS network configured: 169.254.43.1
RNDIS mode started successfully
Starting common services...
Telnet service started
SSH service started
Common services started successfully
USB services started in RNDIS mode    
~ # ifconfig 
usb0      Link encap:Ethernet  HWaddr DE:1A:FC:03:42:32  
          inet addr:169.254.43.1  Bcast:169.254.255.255  Mask:255.255.0.0
          inet6 addr: fe80::dc1a:fcff:fe03:4232/64 Scope:Link
          UP BROADCAST RUNNING MULTICAST  MTU:1500  Metric:1
          RX packets:138 errors:0 dropped:0 overruns:0 frame:0
          TX packets:8 errors:0 dropped:0 overruns:0 carrier:0
          collisions:0 txqueuelen:1000 
          RX bytes:12819 (12.5 KiB)  TX bytes:988 (988.0 B)
\end{lstlisting}

\noindent 在\code{/root/init/usb.conf}中定义了如下的与\text{RNDIS}有关的配置信息:

\begin{lstlisting}[language=sh]
# RNDIS Network Configuration
RNDIS_IP="169.254.43.1"
RNDIS_NETMASK="255.255.0.0"
RNDIS_GATEWAY="169.254.43.10"

# Service Configuration
ENABLE_TELNET="1"
ENABLE_SSH="1"
\end{lstlisting}

\noindent 在开启\text{RNDIS}功能,并开启\text{Telnet}或\text{SSH}服务之后,可以通过上位机连接到板端。默认的用户名为\code{root},默认密码为\code{cdjp123}。同时,板端提供\code{tftp}和\code{sftp}功能,均可用于文件传输。

\textcolor{red}{注意},使用\text{RNDIS}功能需要提前安装相应的驱动,参考为旌文档《\text{Linux}外设驱动开发参考》中第\text{41}页(\text{6-33})。

切换至\text{Serial}模式:

\begin{lstlisting}[language=sh]
~ # /root/init/s50_usb.sh serial
Loading configuration from /root/init/usb.conf
Configuration loaded successfully
Stopping all USB services...
Unloading module: g_ether
Unloading module: usb_f_ecm
Unloading module: usb_f_rndis
Unloading module: u_ether
Unloading module: libcomposite
All USB services stopped
Starting USB Serial (Console) mode...
Loading module: libcomposite
Module libcomposite loaded successfully
Loading module: u_serial
Module u_serial loaded successfully
Loading module: usb_f_acm
Module usb_f_acm loaded successfully
All Serial modules loaded successfully
Setting up USB Serial Gadget...
Found UDC device: fa00000.dwc3
Mounting configfs...
USB Serial Gadget activated successfully
Setting up serial console...
Serial device /dev/ttyGS0 detected
Serial console started on /dev/ttyGS0
Serial mode started successfully
Starting common services...
Telnet service started
SSH service started
Common services started successfully
USB services started in Serial mode

Usage:
    - Connect USB cable to PC
    - Install USB Serial driver if needed
    - Open serial terminal (PuTTY, screen, etc.) at 115200 baud
\end{lstlisting}

\noindent 在开启\text{Serial}模式之后,可以在上位机通过串口连接到板端的串口终端,设置为:

\begin{enumerate}
    \item 波特率:\text{115200}
    \item 数据位:\text{8}
    \item 停止位:\text{1}
    \item 无奇偶校验、无控制流
\end{enumerate}

\noindent 默认的用户名为\code{root},默认密码为\code{cdjp123}。

切换至\text{MTP}模式:

\begin{lstlisting}[language=sh]
~ # /root/init/s50_usb.sh mtp
Loading configuration from /root/init/usb.conf
Configuration loaded successfully
Stopping all USB services...
Removing Serial gadget...
Unloading module: usb_f_acm
Unloading module: libcomposite
All USB services stopped
Starting USB MTP (Media Transfer Protocol) mode...
Module libcomposite is already loaded, skipping
Loading module: usb_f_fs
Module usb_f_fs loaded successfully
Loading module: usb_f_mtp
Module usb_f_mtp loaded successfully
All MTP modules loaded successfully
Creating uMTPrd configuration...
uMTPrd configuration created successfully
Setting up USB MTP Gadget...
Found UDC device: fa00000.dwc3
MTP USB Gadget configured successfully
Starting uMTPrd daemon...
[uMTPrd - 00:23:08 - Info] uMTP Responder
[uMTPrd - 00:23:08 - Info] Version: v1.6.9 compiled Sep 11 2025@11:42:40
[uMTPrd - 00:23:08 - Info] (c) 2018 - 2025 Viveris Technologies
[uMTPrd - 00:23:08 - Info] Using config file /etc/umtprd/umtprd.conf
[uMTPrd - 00:23:08 - Info] Add storage VS839M2 Device - Root Path: /mnt/storage - UID: -1 - GID: -1 - Flags: 0x00000000
[uMTPrd - 00:23:08 - Info] USB Device path : /dev/ffs-mtp/ep0
[uMTPrd - 00:23:08 - Info] USB In End point path : /dev/ffs-mtp/ep1
[uMTPrd - 00:23:08 - Info] USB Out End point path : /dev/ffs-mtp/ep2
[uMTPrd - 00:23:08 - Info] USB Event End point path : /dev/ffs-mtp/ep3
[uMTPrd - 00:23:08 - Info] USB Max packet size : 0x200 bytes
[uMTPrd - 00:23:08 - Info] USB Max write buffer size : 0x2000 bytes
[uMTPrd - 00:23:08 - Info] USB Max read buffer size : 0x2000 bytes
[uMTPrd - 00:23:08 - Info] Read file buffer size : 0x100000 bytes
[uMTPrd - 00:23:08 - Info] Manufacturer string : YourCompany
[uMTPrd - 00:23:08 - Info] Product string : VS839M2
[uMTPrd - 00:23:08 - Info] Serial string : 123456789ABCDEF
[uMTPrd - 00:23:08 - Info] Firmware Version string : 1.0.0
[uMTPrd - 00:23:08 - Info] MTP Exstensions string : 
[uMTPrd - 00:23:08 - Info] Interface string : MTP
[uMTPrd - 00:23:08 - Info] USB Vendor ID : 0x0E8D
[uMTPrd - 00:23:08 - Info] USB Product ID : 0x201D
[uMTPrd - 00:23:08 - Info] USB class ID : 0x06
[uMTPrd - 00:23:08 - Info] USB subclass ID : 0x01
[uMTPrd - 00:23:08 - Info] USB Protocol ID : 0x01
[uMTPrd - 00:23:08 - Info] USB Device version : 0x3008
[uMTPrd - 00:23:08 - Info] USB FunctionFS Mode
[uMTPrd - 00:23:08 - Info] Wait for connection : 0
[uMTPrd - 00:23:08 - Info] Loop on disconnect : 0
[uMTPrd - 00:23:08 - Info] Show hidden files : 1
[uMTPrd - 00:23:08 - Info] File creation umask : System default umask
[uMTPrd - 00:23:08 - Info] Default UID : 0
[uMTPrd - 00:23:08 - Info] Default GID : 0
[uMTPrd - 00:23:08 - Info] inotify : yes
[uMTPrd - 00:23:08 - Info] Sync when close : no
uMTPrd started with PID: 627
MTP USB Gadget activated successfully
MTP mode started successfully
Starting common services...
Telnet service started
SSH service started
Common services started successfully
USB services started in MTP mode

Usage:
    - Connect USB cable to PC
    - PC should detect MTP device automatically
    - Access files through Windows Explorer or MTP client
    - Storage location: /mnt/storage    
\end{lstlisting}

\noindent 在\code{/root/init/usb.conf}中定义了如下的与\text{MTP}有关的配置信息:

\begin{lstlisting}[language=sh]
# USB Device Descriptors - MTP Mode
MTP_VENDOR_ID="0x0e8d"
MTP_PRODUCT_ID="0x201d"
MTP_MANUFACTURER="YourCompany"
MTP_PRODUCT="VS839M2 MTP Device"
MTP_SERIAL_NUMBER="123456789ABCDEF"

# MTP Storage Configuration
MTP_STORAGE_PATH="/mnt/storage"
MTP_DEVICE_NAME="VS839M2 Device"
\end{lstlisting}

\subsubsection{I2C}

\text{Busybox}自带了\code{i2cdetect}与\code{i2ctransfer}工具用于调试\text{I2C}接口。其用法如下:

\begin{lstlisting}[language=sh]
~ # i2cdetect -l
i2c-10	i2c       	Visinextek HDMI                 	I2C adapter
i2c-1	i2c       	Synopsys DesignWare I2C adapter 	I2C adapter
i2c-2	i2c       	Synopsys DesignWare I2C adapter 	I2C adapter
i2c-0	i2c       	Synopsys DesignWare I2C adapter 	I2C adapter
i2c-5	i2c       	Synopsys DesignWare I2C adapter 	I2C adapter
~ # i2cdetect -yr 5
        0  1  2  3  4  5  6  7  8  9  a  b  c  d  e  f
00:          -- -- -- -- -- -- -- -- -- -- -- -- -- 
10: -- -- -- -- -- -- -- -- -- -- -- -- -- -- -- -- 
20: -- -- -- -- -- -- -- -- -- -- -- -- -- -- -- -- 
30: -- -- 32 -- -- -- -- -- -- -- -- -- -- -- -- -- 
40: -- -- -- -- -- -- -- -- -- -- -- -- -- -- -- -- 
50: -- -- -- -- -- -- -- -- -- -- -- -- -- -- -- -- 
60: -- -- -- -- -- -- -- -- -- -- -- -- -- -- -- -- 
70: -- -- -- -- -- -- -- --                         
~ # i2ctransfer -y 5 w2@0x32 0x00 0x0c r1
\end{lstlisting}

\noindent 值得注意的是,在使用\code{i2cdetect}的时候,需要添加\code{-r}参数。

\subsubsection{SPI}

\text{Busybox}本身没有提供任何相关的\text{SPI}工具,如果需要使用命令行的方式快速调试\text{SPI}接口。

在\code{/root/app/utils}下有两个如此工具:

\begin{lstlisting}[language=sh]
~ # ls -la /root/app/utils/spitransfer*
-rwxrwxrwx    1 root     root        136184 Sep  5  2025 /root/app/utils/spitransfer
-rwxrwxrwx    1 root     root         18920 Sep  5  2025 /root/app/utils/spitransfer_GNU    
\end{lstlisting}

\noindent 两者均以相同的方式调用:

\begin{lstlisting}[language=sh]
~ # /root/app/utils/spitransfer --help
Usage:
    Read:  /root/app/utils/spitransfer [spi_dev] [csn] [spi_mode] [reg_addr] -r [bytes]
    Write: /root/app/utils/spitransfer [spi_dev] [csn] [spi_mode] [reg_addr] -w [data_bytes] [data1] [data2] ...

Examples:
    /root/app/utils/spitransfer 1 0 3 0x30 -r 2              # Read 2 bytes from 8-bit register 0x30
    /root/app/utils/spitransfer 1 0 3 0x3040 -r 2            # Read 2 bytes from 16-bit register 0x3040
    /root/app/utils/spitransfer 1 0 3 0x30 -w 2 0x11 0x22    # Write 2 bytes (0x11,0x22) to register 0x30
    /root/app/utils/spitransfer 1 0 3 0x3040 -w 2 0x11 0x22  # Write 2 bytes (0x11,0x22) to 16-bit register 0x3040

SPI Modes:
    0: CPOL:0, CPHA:0
    1: CPOL:0, CPHA:1
    2: CPOL:1, CPHA:0
    3: CPOL:1, CPHA:1

Register Address:
    - 8-bit:  0x00 to 0xFF
    - 16-bit: 0x0100 to 0xFFFF    
\end{lstlisting}

区别在于,\code{spitransfer}使用\code{vs\_regtools}实现,需要在加载\code{vs\_regtools.ko}模块后使用,开机脚本\code{s10_pwr.sh}中默认包含加载了该选项;而\code{spitransfer_GNU}则使用\code{spidev}实现,但需要注意的是,仍然需要在加载\code{vs\_regtools.ko}模块后使用,否则可能出现内核错误。

\subsubsection{UART}

当前的系统中有如下的两个串口:

\begin{enumerate}
    \item \code{/dev/ttyAMA0}
    \item \code{/dev/ttyAMA6}
\end{enumerate}

\noindent 其中\code{/dev/tty/AMA6}用于同\text{MCU}通讯,可以通过\code{echo}、\code{cat}等方式对其进行操作。同时,系统中提供了如下的应用程序用于监听串口消息:

\begin{lstlisting}[language=sh]
~ # /root/app/utils/uart_listen --help
Usage: /root/app/utils/uart_listen <baudrate> <device>
~ # /root/app/utils/uart_listen 115200 /dev/ttyAMA6
Received data: 24 0F 03 00 4A 4C 0A FE DA 
Received data: 24 0F 03 00 4A 4C 0A FE DA 
Received data: 24 0F 03 00 4A 4C 0A FE DA 
Received data: 24 0F 03 00 4A 4C 0A FE DA 
\end{lstlisting}

\subsubsection{GPIO}

\text{GPIO}目前使用集成控制的驱动统一管理,于目录\code{/sys/devices/platform/gpio-controller}下提供了如此的接口:

\begin{lstlisting}[language=sh]
~ # ls -la /sys/devices/platform/gpio-controller/
total 0
drwxr-xr-x    3 root     root             0 Jan  1 00:00 .
drwxr-xr-x   11 root     root             0 Jan  1 00:00 ..
lrwxrwxrwx    1 root     root             0 Jan  1 00:11 driver -> ../../../bus/platform/drivers/gpio-controller
-rw-r--r--    1 root     root          4096 Jan  1 00:11 driver_override
-r--r--r--    1 root     root          4096 Jan  1 00:11 modalias
lrwxrwxrwx    1 root     root             0 Jan  1 00:11 of_node -> ../../../firmware/devicetree/base/gpio-controller
-rw-r--r--    1 root     root          4096 Jan  1 00:00 panel_reset
drwxr-xr-x    2 root     root             0 Jan  1 00:11 power
-rw-r--r--    1 root     root          4096 Jan  1 00:11 regulator_1v8
-rw-r--r--    1 root     root          4096 Jan  1 00:11 regulator_2v8
-rw-r--r--    1 root     root          4096 Jan  1 00:11 regulator_3v3
-rw-r--r--    1 root     root          4096 Jan  1 00:11 sensor_reset
lrwxrwxrwx    1 root     root             0 Jan  1 00:11 subsystem -> ../../../bus/platform
-rw-r--r--    1 root     root          4096 Jan  1 00:11 uevent
-rw-r--r--    1 root     root          4096 Jan  1 00:11 ullc_stdby    
\end{lstlisting}

\noindent 它们分别是:

\begin{enumerate}
    \item \code{regulator\_1v8},用于板端\text{1.8v}供电
    \item \code{regulator\_2v8},用于板端\text{2.8v}供电
    \item \code{regulator\_3v3},用于板端\text{3.3v}供电
    \item \code{sensor\_reset},用于控制摄像头或\text{FPGA}的\text{reset}引脚
    \item \code{panel\_reset},用于控制屏幕的\text{reset}引脚
    \item \code{ullc\_stdby},用于控制\text{ullc\_stdby}引脚
\end{enumerate}

\noindent 举例来说,其使用方法如下:
\begin{lstlisting}[language=sh]
echo off > /sys/devices/platform/gpio-controller/panel_reset
echo 1 > /sys/devices/platform/gpio-controller/panel_reset
\end{lstlisting}

\noindent 保留了两种不同的风格用于控制GPIO的电平。

\subsubsection{旋转编码器}

板端配有型号为\text{RE08115MX}的旋转编码器,系统为其提供了如下的接口:

\begin{enumerate}
    \item 单击
    \item 双击
    \item 正转
    \item 反转
\end{enumerate}

\noindent 该设备集成于\code{/dev/input/event0}中,同时提供了一个事件监听程序用于验证其基本功能,使用方式和日志如下:

\begin{lstlisting}[language=sh]
~ # /root/app/demo/event0_re08115mx 
RE08115MX Rotary Encoder Event Reader
=====================================
Device: /dev/input/event0
Press Ctrl+C to exit

=== Device Information ===
Device name: re08115mx
Device ID: bustype 6, vendor 0001, product 0001, version 0100
=============================

=== Supported Event Types ===
    EV_SYN (0)
    EV_KEY (1)
    EV_REL (2)
==============================

Waiting for events... (Ctrl+C to exit)
Time		Type		Code		Value		Description
----		----		----		-----		-----------
154.566291	EV_REL	8		-1		REL_WHEEL (Rotation) -> Counter-clockwise Rotation (1 steps)
154.566291	EV_SYN	0		0		Event separator
154.619259	EV_REL	8		-1		REL_WHEEL (Rotation) -> Counter-clockwise Rotation (1 steps)
154.619259	EV_SYN	0		0		Event separator
154.652600	EV_REL	8		-1		REL_WHEEL (Rotation) -> Counter-clockwise Rotation (1 steps)
154.652600	EV_SYN	0		0		Event separator
155.326068	EV_REL	8		1		REL_WHEEL (Rotation) -> Clockwise Rotation (1 steps)
155.326068	EV_SYN	0		0		Event separator
155.395305	EV_REL	8		1		REL_WHEEL (Rotation) -> Clockwise Rotation (1 steps)
155.395305	EV_SYN	0		0		Event separator
155.482870	EV_REL	8		1		REL_WHEEL (Rotation) -> Clockwise Rotation (1 steps)
155.482870	EV_SYN	0		0		Event separator
156.471772	EV_KEY	15		1		KEY_TAB (Double Click) (PRESSED) -> Double Click Detected!
156.471772	EV_KEY	15		0		KEY_TAB (Double Click) (RELEASED)
156.471772	EV_SYN	0		0		Event separator
156.624033	EV_KEY	57		1		KEY_SPACE (Single Click) (PRESSED) -> Single Click Detected!
156.624033	EV_KEY	57		0		KEY_SPACE (Single Click) (RELEASED)    
\end{lstlisting}

同时在\code{/sys/devices/platform/re08115m/}目录下,提供了如此的调试接口:

\begin{enumerate}
    \item \code{debug_state}用于控制内核日志
    \item 当两次点击的时间间隔小于等于\code{double_click_ms}则触发双击事件,否则触发单击事件
    \item \code{swap_direction}交换正反转的方向
    \item 累计\code{wheel_min_steps}次完整的相位变化后触发\textbf{一次}旋转事件
    \item 当触发旋转事件后,超过\code{wheel_timeout_ms}才允许触发相反方向的旋转事件
\end{enumerate}

\noindent 这些接口都可以通过\code{read}和\code{write}来实现读写,也可以在命令行中直接使用\code{cat}和\code{echo}来进行简单调试。

\newpage

% -------------------------------------------------------------------------------------
% Section 3 : EMMC分区
% -------------------------------------------------------------------------------------

\section{EMMC分区}

当前系统采用\text{AB}分区实现,分区大小与地址如下所示:

\begin{lstlisting}
/*
* eMMC A/B Partition Layout:
* +----------+------+----------+
* | vs_aov   | 512K | p1       |
* | env      | 512K | p2       |
* | bootload | 7M   | p3       |
* | misc     | 1M   | p4       |
* +----------+------+----------+
* | dtb_a    | 2M   | p5   [A] |
* | dtb_b    | 2M   | p6   [B] |
* | kernel_a | 20M  | p7   [A] |
* | kernel_b | 20M  | p8   [B] |
* | rootfs_a | 512M | p9   [A] |
* | rootfs_b | 512M | p10  [B] |
* +----------+------+----------+
* | vendor   | 32M  | p11      |
* | app      | 256M | p12      |
* | userdata | Rest | p13      |
* +----------+------+----------+
*/  
\end{lstlisting}

\subsection{\text{xOTA}程序}

\subsubsection{基本介绍}

系统升级分为三个部分:

\begin{enumerate}
    \item \code{Bootloader}:该部分不做\text{AB}分区,可在用户空间通过应用程序\code{xOTA}升级实现
    \item \code{Kernel}:该部分提供\text{AB}分区,并实现了完整的检查与回退机制,同样通过\code{xOTA}升级
    \item \code{Users}:该部分用应用层独立控制,不应由系统层进行\text{EMMC raw data}的读写
\end{enumerate}

当前的\code{xOTA}可以通过如下的方式调用:

\begin{lstlisting}
~ # xOTA --help
Usage: xOTA [OPTIONS]

Options:
    --slot, -s                    Show current slot information
    --upgrade, -u                 Upgrade system to inactive slot
    --dtb, -d <file>              DTB file for upgrade
    --kernel, -k <file>           Kernel file for upgrade
    --rootfs, -f <file>           Rootfs file for upgrade (.img format only)
    --set-label, -l               Set slot label
    --target-slot, -t <A|B>       Target slot (A or B)
    --label, -n <name>            Label name (see below)
    --value, -v <value>           Label value
    --help, -h                    Show this help message

Available labels:
    bootable                      Bootable flag (0 or 1)
    successful                    Successful flag (0 or 1)
    active                        Active flag (0 or 1)
    retry_count                   Retry count (0-255)
    boot_attempts                 Boot attempts counter (0-255)
    last_boot_slot                Last boot slot ('a' or 'b' as ASCII value)

Notes:
    - Fast failover is enabled (default retry count = 1)
    - Only .img files are supported for rootfs. Convert .simg using simg2img.
    - Boot attempts counter is used for automatic failover detection.

Examples:
    xOTA --slot
    xOTA --upgrade -d dtb.dtb -k Image -f rootfs.img
    xOTA --set-label -t A -n active -v 1
    xOTA --set-label -t B -n boot_attempts -v 0
    xOTA --set-label -t A -n last_boot_slot -v 97  # 'a' as ASCII    
\end{lstlisting}

仅推荐应用程序通过如下的方式查看与升级系统:

\begin{lstlisting}
xOTA --slot
xOTA --upgrade -d dtb.dtb -k Image -f rootfs.img
\end{lstlisting}

\noindent 其中,\code{--slot}用于检查当前分区,\code{--upgrade}用于进行升级。

\textcolor{red}{警告},用户程序不应该执行任何有关\text{label}或\text{misc}分区的操作;

\textcolor{red}{注意},编译生成的文件系统镜像为\code{simg}格式,\code{xOTA}目前只实现\code{img}格式的升级,因此需要提前使用\code{simg2img}进行格式转换。

\subsubsection{如何升级}

本小节展示了从用户空间完成\text{OTA}的完整过程。

首先,我们将相应的镜像拷贝到系统中。目录位置、校验方式等应当由用户空间程序完成。假设我们现在于\code{/userdata}下有如此的文件:

\begin{lstlisting}[language=sh]
/userdata # ls -l
total 537368
-rw-r--r--    1 root     root      12787720 Jan  1 00:21 Image
drwxr-xr-x    2 root     root         16384 Jan  1 00:00 lost+found
-rw-r--r--    1 root     root     536870912 Jan  1 00:29 rootfs.img
-rw-r--r--    1 root     root         34170 Jan  1 00:21 vs8x9-ci03.dtb     
\end{lstlisting}

接下来我们通过\code{xOTA}对系统进行在线升级:

\begin{lstlisting}[language=sh]
/userdata # xOTA -u -d vs8x9-ci03.dtb -k Image -f rootfs.img 
Info: Rootfs file format validated (.img)
Info: Rootfs file format validated (.img)
Current slot: A, upgrading slot: B
Flashing DTB to /dev/mmcblk0p6...
Flashing kernel to /dev/mmcblk0p8...
Flashing rootfs to /dev/mmcblk0p10...
Upgrade completed successfully!
Please reboot to switch to slot B
Note: Fast failover enabled (retry count = 1)    
\end{lstlisting}

当提示\code{Upgrade completed successfully!}则说明升级成功,此时重启或等待下次开机则默认进入新系统。在系统启动后,\code{s95_ota.sh}会提示当前的系统状态:

\begin{lstlisting}
=== FINAL SLOT STATUS ===
Current slot: B
Misc partition info:
    Magic: ANDROID_BOOT
    Active slot: _b
    Last boot slot: b

Slot A:
    Bootable: 1
    Successful: 1
    Active: 0
    Retry count: 1
    Boot attempts: 0

Slot B:
    Bootable: 1
    Successful: 1
    Active: 1
    Retry count: 1
    Boot attempts: 0        
\end{lstlisting}

\noindent 此时我们可以看到,系统已经切换到了\text{Slot B},并且\text{Slot A}中标签也是符合预期的。

\subsection{镜像打包与烧写}

\subsubsection{\text{U-Boot}修改内容}

为适配\text{AB}分区,因此新增自定义命令\code{ab_boot}用于引导系统。在修改后的系统中,\text{U-Boot}阶段的环境变量如下:

\begin{lstlisting}[language=sh]
vs8x9> pri
baudrate=115200
bootargs=quiet mem=512M console=ttyAMA0,115200n8 rw rootwait rootfstype=ext4 init=/linuxrc
bootcmd=ab_boot
bootdelay=1
fdt_high=0x22600000
fdtcontroladdr=9fe20d00
stderr=serial@320A000
stdin=serial@320A000
stdout=serial@320A000      
\end{lstlisting}

\noindent 其中\code{ab_boot}完成了分区检测、系统引导、\text{fdt}配置等功能。从用户空间查看的启动参数如下:

\begin{lstlisting}[language=sh]
~ # cat /proc/cmdline 
quiet mem=512M console=ttyAMA0,115200n8 rw rootwait rootfstype=ext4 init=/linuxrc root=/dev/mmcblk0p9 rootfstype=ext4 rw rootwait blkdevparts=mmcblk0:512K(vs_aov),512K(env),7M(bootloader),1M(misc),2M(dtb_a),2M(dtb_b),20M(kernel_a),20M(kernel_b),512M(rootfs_a),512M(rootfs_b),32M(vendor),256M(app),-(userdata)
\end{lstlisting}

\subsubsection{系统打包}

系统不为如下的分区进行镜像打包:

\begin{enumerate}
    \item \code{vendor}
    \item \code{app}
    \item \code{userdata}
\end{enumerate}

\noindent 如果在系统启动后,上述分区未被格式化,则将被\code{s20_mount.sh}格式化为\code{ext2}后挂载。因此,如果想要打包一个包含上述分区的完整镜像,则应当修改\text{xml}文件中的如下配置:

\begin{lstlisting}[language=xml]
<Partition_Info>
	<Part PartitionName="vs_aov" FlashType="emmc" FileSystem="none" Start="0M" Length="512K" SelectFile=".\vs_aov.bin"/>
	<Part PartitionName="u-boot-env" FlashType="emmc" FileSystem="none" Start="512K" Length="512K" Display="HIDDEN"/>
	<Part PartitionName="bootloader" FlashType="emmc" FileSystem="none" Start="1M" Length="7M" SelectFile=".\nsec_package.bin"/>
	<Part PartitionName="misc" FlashType="emmc" FileSystem="none" Start="8M" Length="1M" SelectFile=".\misc_init.bin"/>
	<Part PartitionName="dtb_a" FlashType="emmc" FileSystem="none" Start="9M" Length="2M" SelectFile=".\vs8x9-ci03.dtb"/>
	<Part PartitionName="dtb_b" FlashType="emmc" FileSystem="none" Start="11M" Length="2M" SelectFile=".\vs8x9-ci03.dtb"/>
	<Part PartitionName="kernel_a" FlashType="emmc" FileSystem="none" Start="13M" Length="20M" SelectFile=".\Image"/>
	<Part PartitionName="kernel_b" FlashType="emmc" FileSystem="none" Start="33M" Length="20M" SelectFile=".\Image"/>
	<Part PartitionName="rootfs_a" FlashType="emmc" FileSystem="ext4" Start="53M" Length="512M" SelectFile=".\rootfs_ext4.simg"/>
	<Part PartitionName="rootfs_b" FlashType="emmc" FileSystem="ext4" Start="565M" Length="512M" SelectFile=".\rootfs_ext4.simg"/>
	<Part PartitionName="vendor" FlashType="emmc" FileSystem="ext4" Start="1077M" Length="32M" Display="HIDDEN"/>
	<Part PartitionName="app" FlashType="emmc" FileSystem="ext4" Start="1109M" Length="256M" Display="HIDDEN"/>
	<Part PartitionName="userdata" FlashType="emmc" FileSystem="ext4" Start="1365M" Length="0K" Display="HIDDEN"/>
</Partition_Info>
\end{lstlisting}
    
\noindent 可以看到,上述的分区均为\code{HIDDEN}。以打包\code{app}为例:

\begin{enumerate}
    \item 在\text{Linux}系统中创建一个名为\code{app}的文件夹
    \item 将用户空间应用程序放入其中
    \item 使用\code{dd}命令将其打包为需要的格式
\end{enumerate}

在完成上述打包之后,修改\code{Display="HIDDEN"}为对应的\code{app.img}即可。

同时,该功能适用于制作烧录器镜像。

\subsubsection{应用程序打包}

当前的系统推荐使用\text{deb}安装包的形式进行管理,工程位于\href{http://192.168.110.252:12123/cmos/vs819_LLHM/app_dpkg}{app\_dpkg}。其文件结构如下:

\begin{lstlisting}[language=sh]
.
|-- build.sh
`-- pkg
    |-- app
    `-- DEBIAN
\end{lstlisting}

\noindent 其中:

\begin{enumerate}
    \item \code{build.sh}提供了一键打包功能
    \item \code{pkg/app}下对应板端\code{/app}分区的内容
    \item \code{pkg/DEBIAN}为应用描述文件
\end{enumerate}

打包流程如下:

\begin{enumerate}
    \item 在\code{pkg/app}配置板端文件
    \item 修改\code{pkg/DEBIAN/control}中的应用描述信息
    \item 按照需求修改\code{pkg/DEBIAN/prerm}和\code{pkg/DEBIAN/postinst}中的相关功能
    \item 使用\code{./build.sh}进行打包,如果需要增量包、自增版本号等,请参考\code{--help}中的内容
\end{enumerate}

% -------------------------------------------------------------------------------------
% End Document
% -------------------------------------------------------------------------------------

\end{document}